In order to receive frequent feedback from our supervisor, we plan short weekly iterations. This will also enable us to produce small releases in line with our development method. The outline of the iterations (beyond the first) is as follows:

\paragraph{Iteration 2 (due 12/11)} Extend the system's functionality to deal with the more expressive $\mathcal{ALC}$ logic. Implementation and testing will be of highest priority. Some design will be required to be able to naturally extend the logic of $\mathcal{AL}$ and little time should be spent on deployment.

\paragraph{Iterations 3, 4 and 5 (due 19/11, 26/11 and 3/12)} In these iterations we will extend the system's functionality to other more expressive logics, starting with other forms of Description logic, moving to other types of logics if time permits. As these are high risk elements, it is difficult to predict the exact time line and whether all of these logics can be integrated to the system. We will, however, produce a new release in every iteration, with the priority of extending our system to another form of logic in every iteration. 
Intensive work will be done on implementation, and there will be less focus on design. We will spend a significant amount of time testing, as the correctness of our program is critical to its success.

\paragraph{Iteration 6 (10/12)} Integrate the additional features which include a parser (to allow the user to express concepts easily), outputting our proof representation in a human readable format, and open-source deployment of the entire system. Deployment will be of high priority, some implementation and testing will be needed for the additional features.
Feedback from users will be key in producing our final release.
