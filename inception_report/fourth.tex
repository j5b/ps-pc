In order to receive frequent feedback from tests and user-stories, we plan short weekly iterations. This will also enable us to produce small releases in line with our development method. The outline is as follows:

Iteration 1, due: 5/11: Produce a fully-working, sound and complete, system for the simplest of the descriptive logics, AL. At this stage the system will include only its key components: proof and model construction and proof and model checking.  Design activity will be high to provide a relatively compatible design for later iterations. Implementation, testing will also be worked on but to a lesser extent. Deployment should be done easily and quickly through the existing tools described earlier. 

Iteration 2: due 12/11: Extend the system's functionality to deal with the more expressive ALC logic. Implementation and testing will be of highest priority. Some design will be required to be able to naturally extend the logic of AL and little time should be spent of deployment.

Iteration 3, 4 and 5: due on 19/11, 26/11 and 3/12: In these iterations we will extend the system's functionality to other more expressive logics, starting with other forms of Descriptive logic, moving to logics which can be equipped with coalgebraic semantics, such as classical and probabilistic modal logics. As these are high risk elements, it is difficult to predict the exact time line and whether all of these logics can be integrated to the system. We will attempt to produce a new release every iteration. Intensive work will be done on implementation, and much less on design. Testing will also be done intensively as the correctness of our program is critical to its success.

Iteration 6: due 10/12: Integrate the additional features which include a parser (to allow the user to express formulas with additional connectives), outputting our proof representation in a human readable format, and deployment of the entire system. Deployment will be of high priority, some implementation and testing will be needed for the additional features. At this stage we will attempt to change the system according to user-stories. 
