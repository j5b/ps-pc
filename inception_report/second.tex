A blend of several agile software development methods will be adopted; mainly eXtreme Programming, Scrum and Crystal Clear, which are suitable for this project. Agile software development methods will minimise risk and allow the project to adapt to changes quickly by the use of short iterations and ensuring there is a working product at the end of each iteration, which will be presented to the client (Dirk Pattinson) for feedback.

The test-driven nature of eXtreme Programming will be core part of our project development since the integrity of our code is very important and directly affects whether the key requirements are met. Therefore, test suites for unit testing and test cases well known in the proof-search-writer community will be used to test for correctness and performance. These unit tests will be written before the coding of functions or simultaneously, and will be tested as they are implemented. At the end of each iteration, there will be acceptance testing, these and unit tests must be passed before the executable program is shown to the client. In the last iterations, there will be performance testing by benchmarking, to compare our program with similar software and see how we could improve the performance.

Due to the possible difficulty of forming the algorithms in our project, the reflection in theory and being co-located features of Crystal Clear will be adopted. Thus, all members will have the same level of understanding of the algorithms and proof/model system. For the same reason, priority will be given to having a simple solution with clear code first, a feature of eXtreme Programming, to guarantee the existence of a working version of the program.

Extensive code reviews are essential for maintaining high standards of code and for all team members to understand the code, under eXtreme Programming. Hence the project will be hosted on Google Code, chosen for its features allowing easy submission for code reviews by other team members. In addition, Google Code integrates well with our choice of version control, Mercurial. The benefit of using Mercurial are allowing local commits before pushing the changesets to the repository, with relatively easy-to-use graphical user interface extensions; and it is more flexible than other version controls, for example Git.

From eXtreme Programming, pair programming will be used for overlapping areas and more complex parts of the program; and also where sections of the code, implemented by different team members, meet. This ensures code written by different members will be as compatible as possible. Otherwise, each member will be mainly responsible for a certain section of the program.

As in Scrum, we will be keeping a product backlog, sprint backlog and also a `burn down list' to measure the actual progress and velocity of the overall project and iterations against. This will give each member clear specific tasks to complete that contribute towards meeting the key requirements.

Regular scrum-like meetings will be used when we have a group meeting at least twice a week, focusing on:

\begin{itemize}
\item What has each member done since the last meeting?
\item What will each member do until the next meeting?
\item Any issues that have arisen and how to resolve them, such as technical difficulties.
\end{itemize}

These short meetings will encourage progression of the project and highlight problems early on.

For the first few iterations, there will be many extra meetings as well as the scrum-like meetings, to discuss the possible algorithms and proof rules used and to agree specific design requirements of each feature. This is because it will affect all areas of the program and important for everyone to understand the code under eXtreme Programming and make each member's code as compatible as possible.

From considering Unified Process method, we decided to implement the high risk elements in early iterations so that the most critical key requirements and most difficult aims are achieved first before implementing the features with a lower risk.

Because of the nature of our project we do not require more information technology than what we already have.
