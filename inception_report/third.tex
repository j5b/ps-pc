% Specification of length of, and dates for, first iteration.

Our first iteration started on 22/10/10 and will end after two weeks on 05/11/10.
In the beginning of the iteration all team members acquired a theoretical understanding
of the task by reading journal articles on description logic and proof/model searching.
The tasks for our first iteration are as in the sprint backlog in Table~\ref{sprint}.

% Details mapping of iteration tasks onto group members (-> Sprint backlog).

\begin{table}
  \caption{Sprint backlog.}
  \begin{tabular}{l|l|l}
    %\hline
    %\multicolumn{2}{|c|}{Tasks} \\
    \hline
    \textbf{Task} & \textbf{Owner} & \textbf{To be completed by} \\
    \hline
    specification of concept as haskell data type & group & 29/10/10 \\
    specification of model as haskell data type & group & 29/10/10 \\
    specification of proof as haskell data type & group & 29/10/10 \\
    implementation of proof/model search for $\mathcal{AL}$ & Jannis, Sagie & 29/10/10 \\
    implementation of model checker for $\mathcal{AL}$ & Michal & 05/11/10 \\
    implementation of proof checker for $\mathcal{AL}$ & Ka Wai & 05/11/10
  \end{tabular}
  \label{sprint}
\end{table}

% Identification of potential risks during first iteration.

In order to successfully complete the first iteration everyone needs to have a good
understanding of the underlying theory. We need to identify suitable algorithms and
adapt them to our setup. This might take more time than we expected and we will keep
track of that in our regular meetings.

To minimize this risk we aim to implement the most basic functionality first while
still producing a complete working program in the first iteration. The design will
allow extending this functionality in further iterations.

% Planned progress measure for first iteration.

Our objective for the first iteration is a fully working program that, given an
$\mathcal{AL}$ concept, produces either a model for that concept or a proof showing its
unsatisfiablity. Both proof and model can be checked for correctness by the program.

The progress to achieve this goal is measured by regular meetings (twice a week) where
we will discuss the progress so far, identify problems and determine the next tasks. This
way we will always exactly know how far we are from completing the iteration.
