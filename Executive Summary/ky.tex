
Our program was written mainly in Haskell, with the parser for user input generated using the parser generator, Happy. For the output of models and proofs into more readable format, Graphviz and Latex were used respectively.

Technical challenges encountered include the implementation of caching in the proof/model search component and trying to extend the program to allow users to express specific individuals that must exist and to reason with. These difficulties were tackled in frequent informal meetings, in which we discussed the technical issues arisen and ways to resolve them such as through code reviews and bug tracking.

To minimise risk and allow the project to adapt to changes quickly when difficulties occurred, a mixture of eXtreme programming, Scrum and Crystal Clear software development methods was adopted together with extensive testing methods.

Main features of our development method were the use of test-driven development, extensive code reviews, pair programming for more complex parts of the program, frequent Scrum-like meetings, reflective improvement and co-location of members.

Unit testing was done at all stages of development, which revealed most bugs early on or immediately as code was being implemented or underwent a change. To complement, other testing methods used were use of grey-box and black-box testing, which found several obscure bugs.

Overall progression was hindered by the technical difficulties, causing planned iterations to be delayed by one iteration and the eventual decision to no longer implement the individuals extension. With these exceptions, set tasks for each iteration were completed and the finished program described was achieved, with each component consisting of about 150 lines of code.
