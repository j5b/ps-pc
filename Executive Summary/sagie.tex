
The division of labour between all group members was relatively even, where each member was responsible for an individual part of the project and some parts were written together as a group during meetings. In particular Jannis and Saguy were responsible for the proof/model searcher including the implementaion of caching. Jannis and Ka Wei also build a parser for K formulas for additional tests to be done. Michal was responsible for a signigicant amount of testing as well as the proof checker and Ka Wei also dealt with the model checker. 

In terms of the project evaluation, the poject was sucessfull as all the initial requirments were completed and almost all extensions completed. Additionally, it provided a new interface for those interested in reasoning formally about concepts satisfiability. The program's correctness (both the proof/model searcher and the model and proof checkers) is verified and validated, and involves the fundamental design choice whereby the proof/model searcher and the model/proof checkers are developed independently. Both the checkers and searcher are both thoroughly tested. All of the these tests pass along with further validation techniquessuch as code reviews and \emph{hlint}.

To conclude, the duality between the proof/model searcher and the proof and model checkers, which were developed independently, allows for a much greater confidence for the user. We provide an extensively tested program allowing a user to both search and check the satisfiability of concepts, with multiple modes, which allow for extensive error reports as well as a clear graphical representation. As an extension for this project, more expressive logics can be considered (such as $\mathcal{SHIQ}$ which is simple $\mathcal{ALC}$ plus extended cardinality restrictions and transitive and inverse roles), while keeping the confidence of the user in the program's correctness. We believe that this project could provide a basis for perhaps a more expressive (and hence useful) program while keeping the above mentioned advantages. It could be used in academia (for example in the area of Ontologies and the semantic web) or in industry where formal reasoning about the concepts satisfiability is important. 


