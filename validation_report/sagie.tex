\subsection*{Management Tools}

We hosted our project on Google Code which offered us various development and management tools. As part of the Google Code environment, we used Mercurial Version Control and reported bugs and other issues on the Issue Tracker. We also used the Wiki and Code review options in Google which helped us to provide documentation and communicate our changes efficiently and effectively. The use of Google's environment helped us centralise all the management tools in one place which is easy to access and control.

We also used cabal, a system for building and packaging Haskell libraries and programs, to manage different packages and libraries in our project. 

\subsection*{Management Policies}

The most efficient management policy we used, was the frequent SCRUM like meetings which we held three times a week. Two meetings would be held with the groups members only. In these, we would describe the progress with respect the the set tasks, look at any problems or issues that occurred, and set new tasks for the next meeting. The other meeting would be held with our supervisor in which we discussed next iterations and any issues regarding the code. 

These meeting were key in the way code was changed. Using Mercurial we ensured code changes were coordinated. Code reviews and issues in the Issue tracker were addressed and code was changed accordingly. 

Before submitting code we run the unit tests (as a test suite) to see if any tests broke as a result of our code changes. We built additional unit tests and added them to the test suit. The tests suite was managed using cabal, which allowed an easy way of running the test suite. We also run \emph{hlint} to improve the code quality before committing the changes using Mercurial. 

\subsection*{Knowledge Transfer}

Most knowledge transfer was done through the meetings described above. In these informal meeting we discuss technical and theoretical issues
and scheduled meetings. During these meetings, we solved any problems that has risen
from a gap in our technical or theoretical knowledge. Outside of meetings, we use code
reviews through Google Code to help other group members improve their code and share
our knowledge. The most significant knowledge transfer was made during the meeting with our supervisor, were the more difficult concepts and later iterations were discussed.


