We used several methods in addition to unit tests to validate the program. Although
the unit tests were very useful and helped us find many bugs an early stage, they don't
help with more complex bugs. This is why we made use of \emph{hlint}, \emph{code reviews} and \emph{grey/black box testing} to complement them.

The user interface we chose allows to input concepts in a familiar way for anyone
who has worked with formal logic before (e.g. `$A \& B \rightarrow C$'). We have not systematically
tested this on users, but the users of our program were always able to learn the interface quickly.

We use \emph{hlint}, a program similar to lint, that analyses Haskell programs and
suggests changes, to ensure a good and consistent style throughout the program. This
not only ensured that the source code is easier to read and understand, but also
made programming mistakes less likely.

As mentioned in previous reports, we make extensive use of \emph{code reviews}. These
helped improve the quality of the code and made sure that we would understand
each others code. It also lead to the discovery of some small programming mistakes
in the reviews.

After each significant change we used \emph{grey-box testing} to validate these
changes, i.e. we tried, using our knowledge of the inner workings of the program,
to find possible problems. This sometimes lead to the discovery of small issues,
for example a forgotten case.

Furthermore, we used two different ways of \emph{black-box testing} in addition to
the unit testing described above. We wrote a small program that automatically generates
large test cases, runs the proof/model searcher on them, and then either the proof
or the model checker on the result. We also wrote a parser, so we could read complex
`benchmark concepts' that are available from the proof searcher community, and
test these in the same way. Both these approaches turned out to be very
useful and helped us discover some rather obscure bugs (e.g. bugs that would only
occure if concepts were nested in a very particular and complex way).
These kind of tests can also be understood as \emph{stress testing} of our
program, because the cases are larger than we expect normal user input to be
and are very likely to test concepts that go beyond what users would normally
input.

We integrated these tools into a validation process.
We usually ran \emph{hlint} and unit tests before submitting the code to the
repository. Then we gave code reviews and used grey-box testing to improve
the quality of the code and find bugs that weren't found before. On a regular
basis we ran the black-box tests on the whole program, discovering issues
that were missed in the previous steps. This whole setup gave us high
confidence that the program is working correctly.
