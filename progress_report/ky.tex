For our project, there is very little potential ethical issues. The Haskell programming language has no license restrictions regarding written code and we are making our project open source, making it openly available for use for free. All required software which may be required for the optimal use of the system will be freely available such as \LaTeX{}.

A possible ethical issue is our program may be used in the process of planning malicious activities. However, we will not be able to restrict how our written program will be used in practice since our project is simply constructs proofs/models and checks proofs/models are correct.

The environmental impacts of the system when it is deployed are limited and we have adopted approaches to green computing. Supply of the developed system and required materials to develop the system are electronic and therefore have less environmental impact. The outputs of the system (messages, proofs and graphical models) will only be displayed onscreen and stored electronically.

No additional hardware to the typical computer system will be needed and hence no computing equipment will need to be replaced and wasted as a consequence of users choosing to use our system. Also, Haskell code is relatively shorter than writing the same program in other programming languages, so less storage capacity is needed for the actual program files.

On the other hand, since proof and model searching and checking could potentially take a very long time, it will be important to increase algorithmic efficiency as to improve the energy efficiency of our program. To do this, some time will be spent on optimising code and exploring possibly more efficient algorithms. However, long computation time may be unavoidable to ensure the correctness of the program.

By using Google Code, the environmental impact of the development of the system will be reduced, since comments on code and code reviews are well integrated and works just as well if not better than commenting by hand on printed code. Furthermore, Google does cloud computing, which utilises resources more efficiently, and by using Google Code our project also benefits from this.

Our main form of communication outside of meetings is via email; and reports and logs are directly managed and submitted electronically. This minimises the environmental impact as we have a paperless system and telecommuting can be used to reduce travelling costs to the environment.

Meetings are held on campus, which will have no extra cost as they are fixed costs incurring such as heating and lighting, and when team members are all on site not only to attend these group meetings such as lectures. Hence this reduces the environmental impact of travelling. These meetings are necessary for effective face-to-face communication to discuss and explain complex algorithms or problems, which are very likely to occur in our project.
