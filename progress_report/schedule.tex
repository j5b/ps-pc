A more detailed schedule of the following iterations, with a new release at the end of every iteration, is as follows:
\paragraph{Iteration 2 (start 12/11, due 22/11)} In this iteration, we will deal with the potential problem of cycles in the input global assumptions, which causes complications in the implementation of proof search, model construction and model checking. A solution we are considering for this may include a mechanism for caching. This is dealt with in our next iteration as addressing this potential problem later, after other extensions have been added, has the risk of causing the solution to be more difficult to attain and test.
\paragraph{Iteration 3 (start 22/11, due 26/11)} Extend the system to express a new sort, individuals, in logic; this enables a specific individual to be expressed and reasoned with. We will also compare ABox and satisfaction operators as methods of expressing assertion components of a knowledge base (a set of facts); and choose the one we believe is better suited to our program. Implementing this should be relatively straightforward; however, a substantial amount of testing will be needed to ensure all cases are covered and correct.
\paragraph{Iteration 4 (start 26/11, due 3/12)} If the previous iteration progresses well, we will spend this iteration extending the system to cover counting in logic. Although, this appears to be complex so should do this in an earlier iteration by our software development method, we felt the extensions in the previous iteration are more desirable.
\paragraph{Iteration 5 (start 3/12 due 10/12)} Integrate the additional features which include a parser (to allow the user to express concepts easily), outputting our proof representation in a human readable format, and open-source deployment of the entire system. Deployment will be of high priority, some implementation and testing will be needed for the additional features. Feedback from users will be key in producing our final release.
