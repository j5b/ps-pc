Apart from joining the first two iterations as discussed in previous sections, we don't plan to make further changes to the length of iterations, because it is very important for us to produce a working version every week.

Relative activity levels in the following iterations for design, implementation, testing, and deployment remain as planned in the draft schedule. In the next 3 iterations, there will be intensive work on implementation, and less focus on design. For the last iteration, priority will be given to deployment with some implementation and testing; and  feedback from users will be key in producing our final release.

A more detailed schedule of the following iterations is as follows:
\paragraph{Iteration 2 (start 12/11, due 19/11)} In the draft schedule, we had planned to extend the system's functionality to cover a more expressive logic in this iteration. However, from implementing the first iteration, we discovered there was the potential problem of cycles in the input global assumptions, which causes complications in the implementation of proof search, model construction and model checking. A solution we are considering for this may include a mechanism for caching. This is dealt with in our next iteration as addressing this potential problem later, after other extensions have been added, has the risk of causing the solution to be more difficult to attain and test.
\paragraph{Iteration 3 (start 19/11, due 26/11)} As in the draft schedule, the system will be extended to cover more expressive logic. Extensions will be made to express a new sort, individuals, in logic; this enables a specific individual to be expressed and reasoned with. We will also compare \emph{ABox} and \emph{satisfaction operators} as methods of expressing assertion components of a knowledge base (a set of facts); and choose the one that is better suited to our program. Implementing this should be relatively straightforward; however, a substantial amount of testing will be needed to ensure all cases are covered and correct.
\paragraph{Iteration 4 (start 26/11, due 3/12)} If the previous iteration progresses well, we will spend this iteration extending the system to cover counting in logic as planned in our draft schedule to extend the expressibility. Although, this particular extension appears to be complex and so should be in an earlier iteration by our software development method, we felt the extensions in the previous iteration are more desirable.
\paragraph{Iteration 5 (start 3/12, due 10/12)} Integrate the additional features planned in the draft schedule, which include a parser (to allow the user to express concepts easily), outputting our proof representation in a human readable format, and open-source deployment of the entire system. In addition, we also intend to implement finding the shorted proof and/or smallest model for a given concept.