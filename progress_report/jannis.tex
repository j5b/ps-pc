The requirements are as in the inception report, with a few additions. We gave the following requirements in the inception report:
\begin{itemize}
\item Representation of concepts, models and proofs from description logic
\item Implementation of a model/proof searching algorithm for description logic which is correct and which terminates
\item Implementation of a model checking algorithm for description logic that checks
if a model is correct
\item Implementation of a proof checking algorithm for description logic that checks if a proof is correct
\item Provide a standard output format for generated models and proofs 
\end{itemize}
The following requirements have been added, after we found them to be useful features
in our discussions:
\begin{itemize}
\item Output of a helpful explanation in case a proof or model turns out to be incorrect.
\item Provision of an interface to the model/proof checker that allows users to check their own proof model (i.e. independence of proof/model generation and checking).
\end{itemize}
These changes are supported by our supervisor.

We still plan the same feature extensions, as listed below. The expressibility we want to
add to our program is more specific now:

\begin{itemize}
\item Extend the program to more expressive logics with a known proof/model theory
  \begin{itemize}
    \item Add extensions that allow to refer to specific individuals in the domain and
      specify their properties and relations (using either an \emph{Abox} or \emph{satisfaction operator} approach)
    \item Possibly add an extension that allows to specify the number of relations
  \end{itemize}
\item Implementation of a parser for reading concepts in Description Logic
\item Ability to represent models and proofs in a graphical representation
\item User interface to the program
\item Find the shortest proof and/or smallest model for a given concept
\end{itemize}

Our measure for progress -- regular meetings
where we determine the progress -- has turned out to be very useful and we will continue
measuring our progress in this way. This is one of the useful tools provided by our 
blend of agile software development methods: eXtreme Programming, Scrum and Crystal Clear.
Test-driven development, also adapted from this mix, plays an important part in measuring
progress by giving us confidence that the parts of the program that have been written are
working correctly and can be considered as complete for the respective iteration.

As discussed earlier, we already implemented the
core functionality for the $\mathcal{ALC}$-logic and we are very confident that we will
be able to extend this as planned to the more complex logics mentioned above.
