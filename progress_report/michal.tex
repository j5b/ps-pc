Beside dealing with technical issues, the organisation of our project carefully handles the management of the human ressources involved in the project, to guarantee its success. We perform the following actions to ensure the proper work dynamics between the members of the group: assignement of tasks and roles, regular and frequent meetings, evualation by each team member of the work completed.

These actions are undertaken in order to deal with the following problems:

\begin{itemize}
\item How to handle the different levels and areas of skills of each individual in the group?
\item How to ensure the workload in fairly divided amongst group members?
\item How to ensure a smooth progress in the project?
\item How to guarantee that no group member is caught under performing?
\end{itemize}

There are at least 3 meetings per week, and each of these three meetings are scheduled. Two of them are between the group members, and are scheduled on monday at 1pm and on friday at 4pm where we discussed our progress, we assign new tasks and if necessary resolve any technical or managerial issues. The third meeting is usually held on thursday with our supervisor, beside on discussing the progress of the project, we often discuss during this meeting theoretical issues. Other short meetings are organised if the need for them arise.

The different levels and areas of skills of each individual in the group is delt by rigorously matching tasks and people according to the main area of competence during scheduled meetings. Additionally we enhance a learning process to improve our team members skills. This is done through the means of code review, informal meetings whose purpose is to discuss technical and theoretical issues and scheduled meetings. During these meetings we solve any problem that has risen from a gap in our technical or theoretical knowledge. Outside of meetings we use code reviews to help other group members improve their code and share our knowledge, these code reviews are done through a service on Google Code which allows to provide comments on our code.

To ensure that the workload is fairly divided amongst group members, we start by carefully analysing the goals we are required to meet, in order to derive possible tasks. These analysis of goals are done during our scheduled meetings. The tasks are chosen in such a way that each corresponds to a similar amount of work. For instance we have subdivided our main goals into 3 tasks: creation of a model and proof searcher, creation of a model checker, creation of a proof checker. The first tasks was shared between Jannis and Sagie as it represented a bigger workload, the two other tasks was assigned to Michal and Ka respectively. The tasks are then assigned to group members equally. Additionally to subdivision of goals into tasks and their assignement, we have regular and frequent meetings to assess people progress and handle any technical and/or managerial issues. These meetings have for goal to avoid the creation of delays in the project development, guaranteeing smooth progress.

The regular meetings mentionned above also serve for the purpose of discovering if any team member is under performing. If a group member doesn't put enough effort in the project, the lack of work will be dealt as soon as possible with the person involved.

So far we had little problem handling the diversity of skills and competence of our group members as we possess quite similar Haskell Programming and Logic skills. However we had to carefully analyse how the workload could be divided as the proof and model searcher consists on the same algorithm. We have hence decided to assign the task of implementing the proof and model searcher to two group members. The two other tasks we had - model checker and proof checker - were of similar workload and have been assigned to Michal and Ka respectively. Throughout the first iteration we managed to get a smooth and dynamic progress and no one was under performing.

We manage our human ressources mostly through the means of regular scheduled meetings and frequent informal meetings, but we also monitor our progress and our skills through tools provided on Google Code. Our management of the team members have been quite successful and we expect to continue getting good result. 
