Besides dealing with technical issues, we carefully handle the organisation of our project, i.e, the management of the human resources involved in the project, to guarantee its success. We perform the following actions to ensure the proper work dynamics between the members of the group: assignment of tasks and roles, regular and frequent meetings, evaluation by each team member of the work completed.

These actions are undertaken in order to deal with:

\begin{itemize}
\item Handling the different levels and areas of skills of each individual in the group.
\item Ensuring the workload in fairly divided amongst group members.
\item Ensuring a smooth progress in the project.
\item Guaranteeing that no group member is caught underperforming.
\end{itemize}

Here is the description of our scheduled meetings:

\begin{center}
\begin{tabular}{| c | l |  r |}
\hline
Meeting 1 & Monday 12 pm & Group Members \\ \hline
Meeting 2 & Friday 4 pm & Group Members \\ \hline
Meeting 3 & Thursday 2 pm & Group Members and Supervisor \\  
\hline
\end{tabular}
\end{center}

Beside these scheduled meetings other meeting can be schedule if needed. These meetings have for a goal, to avoid the creation of delays in the project development and to guarantee smooth progress.

So far we had little problem handling the diversity of skills and competence of our group members as we possess quite similar Haskell Programming and Logic skills. However if issues arises from the different levels and areas of skills of each invididual in the group they will be dealt rigorously by matching tasks and people according to their main area of competence during schedule meetings. Additionally, we enhance a learning process to improve our team members' skills. This is done through the means of code review, informal meetings whose purpose is to discuss technical and theoretical issues and scheduled meetings. During these meetings, we solve any problem that has risen from a gap in our technical or theoretical knowledge. Outside of meetings, we use code reviews through Google Code to help other group members improve their code and share our knowledge. 

To ensure that the workload is fairly divided amongst group members, we start by carefully analysing the goals we are required to meet, in order to derive possible tasks. This analysis of goals was done during our scheduled meetings. The tasks are chosen in such a way that each corresponds to a similar amount of work. For instance, we have subdivided our main goals into 3 tasks: creation of a model and proof searcher, creation of a model checker, creation of a proof checker. The first task was shared between Jannis and Saguy as it represented a bigger workload, the two other tasks were assigned to Michal and Ka Wai respectively. 

The regular meetings mentioned above also serve for the purpose of discovering if any team member is underperforming. If a group member doesn't put enough effort in the project, the lack of work will be dealt as soon as possible with the person involved by talking to him and refining the deadlines assigned to him. In a severe case we will consult our supervisor.

We manage our human resources mostly through the means of regular scheduled meetings and frequent informal meetings, but we also monitor our progress and our skills through tools provided on Google Code. Our management of the team members have been quite successful throughout the first iteration we managed to get a smooth and dynamic progress and no one was underperforming. We expect to continue getting good results. 
