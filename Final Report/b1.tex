Our project is focused on reasoning in Description Logic a language
for representing knowlegde.  We will provide a short but technical
definition of what we mean by Description Logic. The logical framework
the software deals with is formally known as $\mathcal{ALC}$
(\textit{Attributive Concept Language with Complements}). It forms the
basis for many other Description Logics, and it can be seen as a more
expressive extension of Proposition Logic and a decidable subset of
First-Order logic.

Like in other logics $\mathcal{ALC}$'s definition is subdivided in two
parts one syntactic (language) and the other semantic (meaning). The
syntax is built upon two sets: A set of atomic concept names (unary
relations) and a set of relation names (binary relations). Those two
sets defines the signature. A signature for instance could be
consisting of the following atoms: ``Book'', ``Film''; and the
following relations: ``Bought'', ``Borrowed''. We will now define what
a concept is. A concept in Description Logic is a statement of truth
and falsity just like formulaes are statements of truth in First-Order
logic.

Anything of the following form is a concept:
\begin{itemize}
\item $\top$ (``truth'', ``tautology''),$\bot$ (``bottom'', ``falsity'') and any atomic concept name $A$ is a concept.
\item If $C$ and $D$ are concepts then $C \sqcap D$ is a concept
  called the intersection (or the conjunction) of $C$ and $D$.
\item If $C$ and $D$ are concepts then $C \sqcup D$ is a concept
  called the union (or the disjunction) of $C$ and $D$.
\item If $C$ is a concept then $\neg C$ is a concept called the complement (or the
  negation) of $C$.
\item If $C$ is a concept and $R$ a relation name then $\forall R . C$
  and $\exists R . C$ is a concept called each respectively the
  universal and existential restriction of $C$.
\end{itemize}

Nothing else is a concept. Examples of concepts are $\forall Link
. (Website)$ or $\exists Friend . (\neg Facebook)$. The 
formal semantic for $\mathcal{ALC}$ is defined the notion of
interpretation which gives meaning for Description Logic concepts. An
interpretation is a pair $(\Delta^{I},.^{I})$, where $\Delta^{I}$ is a
non-empty set called the domain and $.^{I}$ is a function sending each
atomic concept to a subset of $\Delta^{I}$ and each relation name to a
subset of $\Delta^{I} \times \Delta^{I}$ and satisfying the following 
properties:

\begin{itemize}
\item $\top^{I} = \Delta^{I}$ and $\bot^{I} = \emptyset$.
\item $(C \sqcap D)^{I} = C^{I} \cap D^{I}$.
\item $(C \sqcup D)^{I} = C^{I} \cup D^{I}$.
\item $(\neg C)^{I} = \Delta^{I} \backslash C^{I}$.
\item $(\forall R . C)^{I} = \{x \in \Delta^{I} \text{: forall } (x,y) \in R^{I} \text{ implies } y \in C^{I}\}$ 
\item $(\exists R . C)^{I} = \{x \in \Delta^{I} \text{: exists } (x,y) \in R^{I} \text{ implies } y \in C^{I}\}$ 
\end{itemize}

An intuitive way to think about $\mathcal{ALC}$, is that we are given
a set of properties an individual might satisfy or not and a set of
possible relations between them. Description Logic in $\mathcal{ALC}$
provides a way to give a representation of knowledge about the
individuals, properties and relations.  $\top$ and $\bot$ are
statements that are either always true or respectively always false
for each individual.  $\sqcap$, $\sqcup$ and $\neg$ can be read each
as ``and'', ``or'' and ``not''. $\forall$ and $\exists$ differs a bit
from First-Order logic in the sense $\forall R$ means every
``R-successors'' and $\exists R$ means a ``R-successors''. For
instance $\forall Link . Website$ could be read as ``every link is to
a website'' and $\exists Friend . (\neg Facebook)$ could be read as
``some friend is not on facebook''

A knowledge base in Description Logic is defined by two collections of
concepts for some signature in $\mathcal{ALC}$. The first one we will
refer as \textit{Gamma} and the second one as the \textit{Givens}. We
say a knowledge base is satisfiable if there exists an interpretation
or model that makes every statement in \textit{Gamma} true at each
point of $\Delta^{I}$ and every statement in \textit{Givens} true at
some point in $\Delta^{I}$. If no such model exists we say a knowledge
base is unsatisfiable. For example if \textit{Givens} consists of one
concept namely $\exists R. \top$ and if \textit{Gamma} consists of
also one concept only $A$ (an atomic concept name), then this
knowledge base is satisfiable since the intrepretation $\Delta^{I}
= \{1\}$ with $A^{I} = \{1\}$ and $R^{I} = \{(1,1)\}$ is a model.
