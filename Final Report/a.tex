
The main goal of our project was to create a tool for formal reasoning
in a knowledge representation framework called Description Logic. More
specifically a solver for Description Logic. Description Logic find
many uses in Artificial Intelligence, Bioinformatics and Web
Semantic. The latter is one of the most prominent examples of
application of Description Logic since it provides a way create a
knowledge domain for websites, and hence has potential to improve the
``surfing on the web'' experience.

More specifically our software provides a tool for showing
satisfiability or otherwise unsatisfiability in Description Logic by
generating a model (for satisfiability) or a proof (otherwise). The
program was written in Haskell for the GHC (Glasgow Haskell Compiler)
Platform and is intented to run on most platforms. It has been tested
on Windows, Mac OS X and Linux and is licensed under the open-source
license GPL 3.0. Additionally it requires Graphivz, Latex, Happy and
Cabal to be installed. 

Unlike most other Description Logic solvers which are focused on Web
Semantics our software possess the following aspects and qualities: 

\begin{itemize}
\item It unifies both the proof and model generator in one tool and
  the generated output are then checked with a seperate function (a
  proof checker or a model checker) to provide confidence to the
  user about the correctness of the output.
\item It assumes the Open World assumption unlike for instance Prolog
\item It is targeted for academia since the program generates output
  that can be easily integrated into latex documents.
\item It is highly flexible especially since the software has been
  written in the high level language Haskell which is particularly
  well suited for Description Logic.
\item It is open-source under the license GPL 3.0 so it can be
  incorporated in other project, it can be forked if necessary and can
  be used to study the algorithms used for deciding satisfiability or
  otherwise unsatisfiability.
\item it focuses on providing short, correct and clear output of the
  obtained models and proofs in a graphical or a textual format, thus
  the software can help study the theory of Description Logic.
\end{itemize}

